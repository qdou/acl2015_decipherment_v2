\section{Decipherment Model}

In this section, we describe the previous decipherment framework that we build on.  This framework follows \newcite{ravi-knight:2011}, who built an MT system using only non-parallel data for translating movie subtitles; \newcite{Dou:2012} and \newcite{Nuhn:2012}, who scaled decipherment to larger vocabularies; and \newcite{dou-knight:2013:EMNLP}, who improved decipherment accuracy with dependency relations between words. 

Throughout this paper, we use $f$ to denote target language or ciphertext tokens, and $e$ to denote source language or plaintext tokens. Given ciphertext $\mathbf{\cipher}:f_{1}...f_{n}$, the task of decipherment is to find a set of parameters $P(f_{i}|e_{i})$ that convert $f$ to sensible plaintext. The ciphertext $\mathbf{\cipher}$ can either be full sentences \cite{ravi-knight:2011,Nuhn:2012} or simply bigrams \cite{dou-knight:2013:EMNLP}. Since using bigrams and their counts speeds up decipherment, in this work, we treat $\mathbf{\cipher}$ as bigrams, where $ \mathbf{\cipher} = \{ \mathbf{\cipher}^n \}_{n=1}^{N} = \{ \cipher_1^n,\cipher_2^n \}_{n=1}^{N} $. 

Motivated by the idea from \newcite{Weaver:1955}, we model an observed cipher bigram $\mathbf{\cipher}^n$ with the following generative story:

\begin{itemize}
\item  First, a language model $P(\mathbf{\plain})$ generates a sequence of two plaintext tokens $e_{1}^n,e_{2}^n$ with probability $P(e_{1}^n,e_{2}^n)$.
\item  Then, substitute $\plain_{1}^n$ with $\cipher_{1}^n$ and $\plain_{2}^n$ with $\cipher_{2}^n$ with probability $P(\cipher_{1}^n \mid e_{1}^n) \cdot P(f_{2}^n \mid e_{2}^n)$.
\end{itemize}

Based on the above generative story, the probability of any cipher bigram $\mathbf{\cipher^n}$ is:
%
\[
\label{p_cipher}
P(\mathbf{\cipher}^n) =  \sum_{e_{1} e_{2}} P(e_{1}e_{2}) \prod_{i=1}^{2}P(f_{i}^n \mid e_{i})
\]
%

%If the entire ciphertext corpus contains $N$ such bigrams $F_{1}...F_{N}$, we write down the probability of the ciphertext corpus as:
The probability of the ciphertext corpus,
%
\[
\label{p_corpus}
P( \{ \mathbf{\cipher}^n \}_{n=1}^{N} ) =  \prod_{n=1}^{N} P(\mathbf{\cipher}^{n})
\]
%

There are two sets of parameters in the model: the channel probabilities $\{ P(\cipher \mid \plain) \} $ and the bigram language model probabilities $\{ P(\plain' \mid \plain) \} $, where $\cipher$ ranges over the ciphertext vocabulary and $\plain,\plain'$ range over the plaintext vocabulary. Given a plaintext bigram language model, the training objective is to learn $P(\cipher \mid \plain)$ that maximize $P( \{ \mathbf{\cipher}^n \}_{n=1}^{N} )$. When formulated like this, one can directly apply EM to solve the problem \cite{knight-EtAl:2006}. However, EM has time complexity $O( N\cdot V_{e}^{2})$ and space complexity $O(V_{f}\cdot V_{e})$, where $V_{f}$, $V_{e}$ are the sizes of ciphertext and plaintext vocabularies respectively, and $N$ is the number of cipher bigrams. This makes the EM approach unable to handle long ciphertexts with large vocabulary size. 
%Unfortunately, EM is not scalable when $V_{f}$, $V_{e}$, and $N$ are very large.
%$\cipher \in 1,\ldots,V_{\plain}$ and $\plain, \plain' \in 1,\ldots,V_e$

An alternative approach is Bayesian decipherment \cite{ravi-knight:2011}. We assume that $P(\cipher \mid \plain)$ and $P(\plain' \mid \plain)$ are drawn from a Dirichet distribution with hyper-parameters $\alpha_{\cipher,\plain}$ and $\alpha_{\plain,\plain'}$, that is: 

\begin{align*}
P(\cipher \mid \plain) & \sim Dirichlet(\alpha_{\cipher,\plain}) \\ 
P(\plain \mid \plain') & \sim Dirichlet(\alpha_{\plain,\plain'}).
%P(C=0 \mid x) &= \frac{k q(x)}{\frac{n}Z p(x) + k q(x)} \\
%P(C=1 \mid x) &= \frac{\frac{n}Z p(x)}{\frac{n}Z p(x) + k q(x)}
\end{align*}

The remainder of the generative story is the same as the noisy channel model for decipherment. In the next section, we describe how we learn the hyper parameters of the Dirichlet prior. Given $\alpha_{\cipher,\plain}$ and $\alpha_{\plain,\plain'}$, The joint likelihood of the complete data and the parameters,
\begin{align} \label{joint_likelihood}
&P( \{ \mathbf{\cipher}^n , \mathbf{\plain}^n \}_{n=1}^{N}, \{ P(\cipher \mid \plain) \}, \{ P(\plain \mid \plain') \} )  \notag \\
 &= P( \{ \mathbf{\cipher}^n \mid \mathbf{\plain}^n \}_{n=1}^{N}, \{ P(\cipher \mid \plain) \}) \notag \\
     &P(  \{  \mathbf{\plain}^n \}_{n=1}^{N},P(\plain \mid \plain')) \notag \notag \\
 &= \prod_{\plain}  \frac{\dirgamma{\sum_{\cipher} \alpha_{\cipher,\plain}}} {\prod_{\cipher} \dirgamma{\alpha_{\plain,\cipher}}} \prod_{\cipher} P(\cipher \mid \plain)^{\#(\plain,\cipher)+\alpha_{\plain,\cipher} -1}  \notag \\
  &\prod_{\plain}  \frac{\dirgamma{\sum_{\plain'} \alpha_{\plain,\plain'}}} {\prod_{\plain'} \dirgamma{\alpha_{\plain,\plain'}}} \prod_{\cipher} P(\plain \mid \plain')^{\#(\plain,\plain')+\alpha_{\plain,\plain'} -1} , 
\end{align}

where $\#(\plain,\cipher)$ and $\#(\plain,\plain')$ are the counts of the translated word pairs and plaintext bigram pairs in the complete data, and $\dirgamma{\cdot}$ is the Gamma function. Unlike EM, in Bayesian decipherment, we no longer search for parameters $P(\cipher \mid \plain)$ that maximize the likelihood of the observed ciphertext. Instead, we draw samples from posterior distribution of the plaintext sequences given the ciphertext. Under the above Bayesian decipherment model, it turns out  that the probability of a particular cipher word $\cipher_{j}$ having a value $k$, given the current plaintext word $\plain_j$, and  the samples for all the other ciphertext and plaintext words, $\mathbf{\cipher}_{-j}$ and $\mathbf{\plain}_{-j}$, is:

\begin{equation} \label{prob_bayes_ciphertext}
P(\cipher_j = k \mid \plain_j,\mathbf{\cipher}_{-j}, \mathbf{\plain}_{-j}) = \frac{\#(k, \plain_j)_{-j} + \alpha_{\plain_j,k}}{\#(\plain_j)_{-j}+\sum_{\cipher} \alpha_{\plain_j,\cipher}}.
\end{equation}

Where, $\#(k, \plain_j)_{-j}$ and $\#(\plain_j)_{-j}$ are the counts of the ciphertext, plaintext word pair and plaintext word in the samples excluding $\cipher_j$ and $\plain_j$. Similarly, the probability of a plaintext word $\plain_j$ taking a value $l$ given samples for all other plaintext words, 
\begin{equation} \label{prob_bayes_plaintext}
P(\plain_j = l \mid \mathbf{\plain}_{-j}) = \frac{\#(l, \plain_{j-1})_{-j} + \alpha_{l,\plain_{j-1}}} {\#(\plain_{j-1})_{-j} + \sum_{\plain} \alpha_{\plain,\plain_{j-1}}}.
\end{equation}


%\begin{equation}
%\end{equation}

Since we have large amounts of plaintext data, we can train a high-quality dependency-bigram language model, $P_{LM}(\plain \mid \plain')$ and use it to guide our samples and learn a better posterior distribution. For that, we define $\alpha_{\plain,\plain'} = \alpha P_{LM}(\plain \mid \plain')$, and set $\alpha$ to be very high. The probability of a plaintext word (Equation~\ref{prob_bayes_plaintext}) is now

\begin{equation} \label{prob_bayes_plaintext_lm}
P(\plain_j = l \mid \mathbf{\plain}_{-j}) \approx P_{LM}(l \mid \plain_{j-1}).
\end{equation}

%\marginpar{We have to say that in previous work , $\alpha_{\plain,\cipher}$ has been set to $\alpha * uniform$. Better to reflect that $\alpha_{\plain_j,k}= \alpha * base ?$}. To sample from the posterior, we iterate over the observed ciphertext bigram tokens and use equations~\ref{prob_bayes_ciphertext} and~\ref{prob_bayes_plaintext_lm} to sample a plaintext token with probability

\begin{align} \label{prob_sampling_plaintext}
&P( \plain_j \mid \mathbf{\plain}_{-j}, \mathbf{\cipher} ) \propto  P_{LM}(\plain_j \mid \plain_{j-1})  \\
&           P_{LM}(\plain_{j+1} \mid \plain_{j})   P(\cipher_j \mid \plain_j,\mathbf{\cipher}_{-j}, \mathbf{\plain}_{-j}).
\end{align}

\iffalse
%
\[
\label{p_sample}
P_{sample}(e_{1}e_{2}) =  P(e_{1}e_{2}) \prod_{i=1}^{2}P_{CRP}(f_{i}|e_{i})
\]
%
In the above equation, the translation probability $P_{CRP}(f_{i}|e_{i})$ is modeled by the Chinese Restaurant Process(CRP) as defined in Equation \ref{p_channel}.
%!TEX encoding = UTF-8 Unicode
\[
\label{p_channel}
P_{CRP}(f_{i}|e_{i}) = \frac{\alpha P_0(f_{i}|e_{i})+count(f_{i},e_{i})}{\alpha+count(e_{i})}
\]
%
where $P_{0}$ is a base distribution, also known as a prior, and $\alpha$ is a parameter that controls how much we trust the base distribution. $count(f_{i},e_{i})$ and $count(e_{i})$ record the number of times $f_{i},e_{i}$ and $e_{i}$ appear in previously generated samples respectively. The base distribution is given independently, and in all the previous work, it is set to uniform.
\fi

\iffalse
At the end of sampling, we compute $P(\cipher \mid \plain)$ from ciphertext and its plaintext samples using maximum likelihood estimation:

\[
\label{mlh_estimation}
P(\cipher \mid \plain) =  \frac{\#(\cipher,\plain)}{\#(\plain)}.
\]
\fi

In previous work~\cite{Dou:2012}, the authors use symmetric priors over the channel probabilities, where $\alpha_{\plain,\cipher} = \alpha \frac{1}{V_\cipher}$, and they set $\alpha$ to $1$. Symmetric priors over word translation probabilities are a poor choice, as one would not a-priori expect plaintext words and ciphertext words to cooccur with equal frequency. Bayesian inference is a powerful framework that allows us to inject useful prior information into the sampling process, a feature that we would like to use. In the next section, we will describe how we model and learn better priors using distributional properties of words. In subsequent sections, we show significant improvements over the baseline by learning better priors.


